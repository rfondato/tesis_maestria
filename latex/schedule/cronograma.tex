\documentclass[10pt]{article}

\usepackage[spanish]{babel}  
\usepackage[utf8]{inputenc} 
\usepackage{graphicx}
\usepackage{subfigure}
\usepackage{amsfonts}
\usepackage{comment}
\usepackage{amssymb,amsmath}
\usepackage{graphicx,psfrag,float}
\usepackage[round]{natbib}


\voffset-2.5cm 
\hoffset 0.0cm 
\topmargin 2cm
\oddsidemargin-5mm% 
\evensidemargin 0mm
\textwidth=17cm
\textheight=23.2cm 
%\columnsep=0.8cm
\pagestyle{empty}
\renewcommand{\baselinestretch}{1}  
\sloppy  
\widowpenalty100000   
\clubpenalty100000
\parskip 2mm   


\begin{document}
\title{{\bf Tesis de Maestría}\\ Reconocimiento de acciones humanas en video, utilizando detección de contornos y Deep Learning
}
\author{Cronograma de Trabajo \\
	\ \\
	 \bf{Alumno}: Rodrigo Fondato\\
 \ \\
\bf{Tutor: Juliana Gambini}}
\date{}
\maketitle
	\Large{\textbf{}}
	
\section{Descripción del problema}

El reconocimiento de actividades humanas (HAR, por sus siglas en inglés), es la habilidad para interpretar movimientos y gestos efectuados por personas a través de sensores, para luego determinar qué acciones o actividades están realizando \citep{Ann2014}.

Está técnica tiene múltiples aplicaciones, como la predicción de actividades peligrosas o delictivas a partir de secuencias de video tomadas por cámaras instaladas en la vía pública \citep{Ryoo2011}, aplicaciones médicas, como un sistema para el monitoreo de ejercicios de rehabilitación \citep{Gonzalez-Ortega2014} o la generación de instrucciones para interactuar con un software mediante el reconocimiento de gestos y movimientos de los usuarios \citep{Gerling2012}.

A partir del desarrollo del hardware de aceleración gráfica (GPUs) y la publicación de AlexNet \citep{Krizhevsky2012}, las redes profundas tomaron protagonismo en el campo de visión por computadora. Muchas investigaciones recientes en el campo del reconocimiento de acciones humanas están mayormente basadas en el uso de redes convolucionales, debido a que permiten extraer descriptores robustos en forma automática a partir de los datos de entrenamiento \citep{Ji2013, Feichtenhofer2016, Varol2018, Ullah2017, Zhu2019}.

Sin embargo, requieren una gran cantidad de recursos computacionales (poder de procesamiento y memoria), tiempo y disponibilidad de datos para su entrenamiento, debido a la cantidad de parámetros que deben ser ajustados \citep{Li2016}.

Si bien una ventaja de las redes convolucionales es que la cantidad de parámetros no depende del tamaño de los datos de entrada (ya que los pesos se comparten), algunas investigaciones demuestran que sí existe una correlación entre dicho tamaño y la complejidad óptima de la red, necesitando una mayor cantidad de  capas o filtros al incrementar la resolución de entrada. Esto puede explicarse dado que, a una mayor resolución de imagen, son necesarias más capas para obtener campos receptivos más amplios que permitan capturar características que consideren una cantidad superior de píxeles \citep{Tan2019}.

Para el caso particular del procesamiento de video, a las dos dimensiones temporales de entrada (alto y ancho) se le suma una tercera dimensión, el tiempo, medido en cantidad de cuadros. Para procesar entradas de datos de ésta dimensionalidad, fueron propuestas redes convolucionales que utilizan filtros 3D, denominadas C3D. Debido a la dimensión adicional en los filtros, la cantidad de parámetros de éstas redes pueden escalar exponencialmente, demandando un mayor poder de cómputo \citep{Li2019}.

Para evitar el aumento de la complejidad de la red, es necesario conservar un determinado tamaño en los datos de entrada. Por éste motivo, existe un compromiso entre la resolución de las imágenes de entrada y la cantidad de cuadros consecutivos que serán considerados para la detección de acciones. A modo de ejemplo, uno de los modelos propuestos utiliza solo 9 cuadros consecutivos para efectuar la clasificación \citep{Ji2013}. Un método para aumentar el horizonte temporal es disminuir la resolución de los videos utilizados \citep{Asadi-Aghbolaghi2017}.

En muchas aplicaciones, la detección debe ser realizada en tiempo real. Para éstos casos hay un restricción en el tiempo total disponible para realizar el procesamiento. 

El problema que se intenta resolver es entonces la detección computacionalmente eficiente de acciones humanas básicas en video, mediante el uso de contornos humanos como datos de entrada, permitiendo la reducción de la cantidad de parámetros a ajustar y consecuente complejidad del modelo.

\section{Objetivos específicos}

Se propone realizar las siguientes actividades: 

\begin{enumerate}
	
	\item Definición del proceso de gestión del proyecto, selección de herramientas a utilizar para la gestión y creación de lista de tareas.
	
	\item Definición del proceso de revisión de la literatura y fuentes a investigar, armado de consultas y criterios de selección de artículos, y ejecución del proceso de selección. Lectura de los artículos, y armado del estado de la cuestión.
	
	\item Identificación, clasificación, selección y obtención de bases de datos públicas que contengan segmentos de video que muestren acciones humanas.
	
	\item Análisis y selección de los segmentos de video según categorías, calidad, duración y formato.
	
	\item Implementación y ejecución de un proceso para adaptar y homogeneizar los distintos archivos, obteniendo como resultado una única base de datos.
	
	\item Diseño e implementación del modelo de reconocimiento de acciones humanas.
	
	\item Diseño de las métricas e implementación de la herramienta que las obtendrá.
	
	\item Entrenamiento y optimización del modelo.
	
	\item Revisión del diseño de las características y el modelo, propuesta e implementación de mejoras. Obtención de nuevas métricas y elección del modelo ganador.
	
	\item Instalación y ejecución de modelos pre-entrenados existentes. Obtención de métricas.
	
	\item Análisis de los resultados obtenidos en la comparación de modelos.  Elaboración de conclusiones y finalización del documento de tesis.
	
\end{enumerate}

\pagebreak

\section{Cronograma de trabajo}
%max 1 pag.

El Cronograma de Trabajo se muestra en el siguiente cuadro:

\begin{table}[!ht]
	%\caption{Kurtosis and Skewness}
	\renewcommand{\arraystretch}{1.5}
	\label{tab:kurskew}
	\begin{center}
		\begin{tabular}{| c | c | c | c | c | c | c | c | c | c | c |c |}\cline{1-12}
			\multicolumn{1}{|c|}{Actividad} & \multicolumn{11}{|c|}{Mes}\\ \hline
			\multicolumn{1}{|c|}{ } &	\multicolumn{1}{|c|}{ 1} & 					\multicolumn{1}{|c|}{2} & \multicolumn{1}{|c|}{3} & 		\multicolumn{1}{|c|}{4} & \multicolumn{1}{|c|}{ 5} & 					\multicolumn{1}{|c|}{6} & \multicolumn{1}{|c|}{7} & 		\multicolumn{1}{|c|}{8} & \multicolumn{1}{|c|}{ 9} & 					\multicolumn{1}{|c|}{10} & \multicolumn{1}{|c|}{11} \\ \hline
			Actividad 1 & x & & & & & & & & & & \\ \hline
			Actividad 2 & x & x & x & & & & & & & & \\ \hline
			Actividad 3 & & & & x & & & & & & & \\ \hline
			Actividad 4 & & & & x & x & & & & & & \\ \hline
			Actividad 5 & & & & & x & x & & & & & \\ \hline
			Actividad 6 & & & & & & x & x & x & & & \\ \hline
			Actividad 7 & & & & & & & & x & & & \\ \hline
			Actividad 8 & & & & & & & & & x & & \\ \hline
			Actividad 9 & & & & & & & & & x & x & \\ \hline
			Actividad 10 & & & & & & & & & & x & x \\ \hline
			Actividad 11 & & & & & & & & & & & x \\ \hline
		\end{tabular}
	\end{center}
\end{table}

\pagebreak

\section{Bilbiografía}

\renewcommand\refname{}
\bibliographystyle{apalike} % for APA
\bibliography{bibliography}

\end{document}