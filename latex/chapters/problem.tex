% !TEX encoding = UTF-8 Unicode
% !TEX root = ../thesis.tex
\chapter{Planteamiento del Problema} \label{Problem}

Las técnicas para la detección automática de acciones realizadas por personas en capturas de videos pueden dividirse, a grandes rasgos, en aquellas que utilizan herramientas de procesamiento de imágenes para obtener descriptores espacio-temporales y el posterior uso de modelos clásicos de machine learning para la clasificación, y en aquellas basadas en la explotación de grandes volúmenes de datos y la consecuente utilización de modelos de deep learning para etiquetar las acciones detectadas. \\

Estos últimos modelos, más investigados en la actualidad, permiten obtener descriptores más robustos que los primeros, pero requieren una gran cantidad de recursos computacionales (poder de procesamiento y memoria), tiempo y disponibilidad de datos para su entrenamiento, debido al tamaño de los archivos de video en memoria y la cantidad de parámetros que deben ser optimizados. Existe además, un compromiso entre el horizonte temporal y la resolución espacial. Es decir, a mayor resolución de la imagen de entrada, menor cantidad de cuadros podrán ser analizados por el modelo para efectuar la clasificación, y viceversa. \\

El problema que se intenta resolver es, para una determinada resolución de video de entrada, la expansión del horizonte temporal y la disminución de la cantidad de parámetros de un modelo basado en deep learning, con la consecuente reducción de la cantidad de datos y recursos computacionales necesarios para el entrenamiento, obteniendo a la vez las ventajas de robustez y rendimiento que otorgan este tipo de modelos, específicamente para el ámbito de reconocimiento de acciones humanas básicas.

\section{Formulación del problema: Preguntas de Investigación}

\begin{enumerate}
	\item ¿ Qué bases de datos existen que contengan videos clasificados o no clasificados, muestren a una o más personas realizando distintas acciones y sean de acceso público?
	\begin{enumerate}
		\item ¿Qué formato y calidad presentan los videos? ¿Son todos los segmentos de la misma duración, tamaño, formato y calidad?
		\item En caso de estar clasificados, ¿De qué manera están etiquetados?
	\end{enumerate}
	\item ¿Qué modelos, basados en visión por computadora, son utilizados actualmente para reconocer acciones humanas en capturas de video?
	\begin{enumerate}
		\item ¿Cuáles de ellos se basan en la utilización de machine y/o deep learning?
	\end{enumerate}
	\item ¿Qué técnicas clásicas de procesamiento de imágenes se utilizan para construir descriptores útiles para la detección de actividades humanas y cuáles son las ventajas y limitaciones de cada uno? 
	\begin{enumerate}
		\item ¿Qué modelos existen que combinen éstas técnicas con el uso de deep learning?
	\end{enumerate}
	\item ¿Qué modelos se utilizan actualmente para detectar contornos o siluetas, y que ventajas y limitaciones poseen?
	\begin{enumerate}
		\item ¿Cuáles de éstos métodos se usan para el reconocimiento de actividades humanas?
	\end{enumerate}
	\item ¿Qué técnicas combinan la detección de siluetas o contornos humanos y posterior uso de modelos de deep learning para clasificar actividades humanas en video?
\end{enumerate}

\section{Objetivo de la investigación}

Combinar técnicas de detección de contornos y modelos basados en machine o deep learning para, a partir de capturas de video y en forma automática, clasificar acciones realizadas por personas, etiquetando cada cuadro con la actividad detectada, evaluando el rendimiento, consumo de recursos y tiempos de inferencia, y compararlos con modelos de deep learning que sean utilizados actualmente para este fin.

\subsection{Objetivos Específicos}

\begin{itemize}
	\item Identificar, clasificar, adaptar y unificar bases de datos de acceso público que contengan registros de video donde se observen diferentes actividades realizadas por personas.
	\item Implementar un algoritmo de preprocesamiento que, dadas las capturas de video, permita construir características derivadas de la aplicación de técnicas de detección de contornos y/o segmentación de personas.
	\item Adaptar o combinar modelos que, dadas las características construidas como resultados del objetivo anterior, permitan clasificar las actividades realizadas por personas, en un segmento de video.
	\item Evaluar el desempeño de la herramienta desarrollada o adaptada sobre uno o más sets de datos de testeo, mediante la utilización de métricas para modelos de clasificación, y compararla con el rendimiento de técnicas existentes de deep learning que utilicen directamente las capturas de video como entrada de datos.
\end{itemize}

\subsection{Justificación del estudio}

Las técnicas de detección de acciones humanas en capturas de video que utilizan modelos de deep learning entrenados y aplicados cuadro por cuadro requieren el ajuste de una gran cantidad de parámetros para obtener un rendimiento útil, y por consiguiente necesitan un gran poder de cómputo y una elevada cantidad de datos de entrenamiento para reducir el sobreajuste. La utilización de métodos de detección de contornos y/o segmentación de personas para la construcción de características que puedan ser utilizadas como entrada a un modelo de machine o deep learning, podría reducir este costo. \\

El presente trabajo pretende estudiar el desempeño, uso de recursos y tiempos de inferencia de la técnica propuesta y evaluar ventajas y desventajas al compararlo con la utilización de modelos basados en deep learning empleados actualmente para el mismo fin, otorgando como resultado, métricas concretas que permitan comparar objetivamente ambos enfoques. \\

El modelo resultante permitirá reconocer acciones humanas básicas en video y clasificarlas, en forma automática, utilizando únicamente videos RGB como entrada de datos, lo cual lo hace apto para aplicaciones de video vigilancia, médicas, entre otras.
